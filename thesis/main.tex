% options:
% thesis=B bachelor's thesis
% thesis=M master's thesis
% czech thesis in Czech language
% slovak thesis in Slovak language
% english thesis in English language
% hidelinks remove colour boxes around hyperlinks


\documentclass[thesis=B,czech]{FITthesis}[2012/06/26]

\usepackage[utf8]{inputenc} % LaTeX source encoded as UTF-8

\usepackage{graphicx} %graphics files inclusion
% \usepackage{amsmath} %advanced maths
% \usepackage{amssymb} %additional math symbols

\usepackage{dirtree} %directory tree visualisation



% % list of acronyms
% \usepackage[acronym,nonumberlist,toc,numberedsection=autolabel]{glossaries}
% \iflanguage{czech}{\renewcommand*{\acronymname}{Seznam pou{\v z}it{\' y}ch zkratek}}{}
% \makeglossaries

\newcommand{\tg}{\mathop{\mathrm{tg}}} %cesky tangens
\newcommand{\cotg}{\mathop{\mathrm{cotg}}} %cesky cotangens

% % % % % % % % % % % % % % % % % % % % % % % % % % % % % % 
% ODTUD DAL VSE ZMENTE
% % % % % % % % % % % % % % % % % % % % % % % % % % % % % % 

\department{Katedra softwarového inženýrství}
\title{Systém pro analýzu proudu dat v reálném čase}
\authorGN{David} %(křestní) jméno (jména) autora
\authorFN{Viktora} %příjmení autora
\authorWithDegrees{David Viktora} %jméno autora včetně současných akademických titulů
\supervisor{Ing. Adam Šenk}
\acknowledgements{Poděkování ....}
\abstractCS{V~několika větách shrňte obsah a přínos této práce v~češtině. Po přečtení abstraktu by se čtenář měl mít čtenář dost informací pro rozhodnutí, zda chce Vaši práci číst.}
\abstractEN{Sem doplňte ekvivalent abstraktu Vaší práce v~angličtině.}
\placeForDeclarationOfAuthenticity{V~Praze}
\declarationOfAuthenticityOption{4} %volba Prohlášení (číslo 1-6)
\keywordsCS{Nahraďte seznamem klíčových slov v češtině oddělených čárkou.}
\keywordsEN{Nahraďte seznamem klíčových slov v angličtině oddělených čárkou.}


\begin{document}

% \newacronym{CVUT}{{\v C}VUT}{{\v C}esk{\' e} vysok{\' e} u{\v c}en{\' i} technick{\' e} v Praze}
% \newacronym{FIT}{FIT}{Fakulta informa{\v c}n{\' i}ch technologi{\' i}}

\begin{introduction}
	Kratce o jednotlivých bodech zadání a struktuře prace, motivace
	
\end{introduction}

%\chapter
%\section
%\subsection
%


\chapter{Úvod do problematiky}
	V současné době generujeme obrovská množství dat - podle některých odhadů to například v roce 2012 mohlo být až 2,5 exabajtů za den\cite{citace1}. Od té doby se rychlost přibývání dat stále zvyšuje. Například nárůst objemu dat dostupných na internetu je způsoben stále větší dostupností internetu, ke kterému je v roce 2016 připojeno již XXX obyvatel planety\cite{citace2}. Roste také míra využívání internetu. Oproti dřívějšku na internetu trávíme stále více času a využíváme například sociální sítě, cloudová úložiště nebo další online služby. Také ve firemní i státní sféře výrazně roste stupeň využívání informačních technologií a v návaznosti na to objem produkovaných dat. 

	S rostoucím objemem dat obecně nastává problém s jejich zpracováním, a to především se zpracováním v dostatečně krátkém, ideálně reálném, čase. Klasické technologie používané v minulosti se stále častěji ukazují jako nedostatečně rychlé a je proto nutné sáhnout po nových technologiích určených pro práci s velkými objemy dat. Touto problematikou obecně se zabývá relativně nový a často skloňovaný obor Big Data. Bez technologií pro Big Data se v dnešní době neobejdou především internetové vyhledávače nebo sociální sítě\cite{4}. Veliké možnosti jejich využití se ale nabízejí i v dalších oblastech a jejich rozvoj je předpokladem pro rozšíření tzv. Internet of Things\cite{3}. O Big Datech hovoříme v případech, ve kterých je potřeba zpracovávat objemy dat v řádech gigabajtů a více. 
	
	Tyto technologie je samozřejmě možné použít jak na data strukturovaná, tak především na ta nestrukturovaná. Jelikož je těch nestrukturovaných až XXXX krát více než těch strukturovaných\cite{3}, a protože je jejich zpracování zpravidla výpočetně náročnější, jsou právě technologie pro Big Data vhodnou volbou. Klasickým příkladem nestrukturovaných dat je lidská řeč v psané formě - ta rozhodně obsahuje spoustu informací, ale ve formě kterou je počítačově složité analyzovat. Jako příklad takových dat můžeme uvést právě příspěvky na sociální síti Twitter, kterými se tato práce zabývá. 
	
	Analýzou těchto dat se zabývají obory jako Data Mining či Natural Language Processing. Pomocí nich je často možné získat opravdu cenné informace. Například analýzou dat pohybu uživatele po webové stránce a jeho chováním můžeme odhalit nedostatky tohoto webu a jejich odstraněním zvýšit míru konverze. Hovoříme-li v kontextu sociální sítě Twitter, možnosti jsou ještě zajímavější. Na základěji příspěvků a vyplněných informací jednotlivých uživatelů můžeme například odhadovat jejich volební preference nebo nabízet velice přesně cílenou reklamu. 

\section{Big data a technologie pro práci s nimi}
	V předchozím textu byl již několikrát zmíněn pojem Big Data s vysvětlením, že se jedná o data o velikosti jednotek gigabajtů a více. To je ale velice zjednodušený popis tohoto termínu. Přesná definice však neexistuje a na celý problém se dá dívat různými způsoby. Jedna z dalších definic říká, že o Big Datech mluvíme v případě, kdy klasické databázové a softwarové nástroje kvůli objemu těchto dat selhávají\cite{http://www.webopedia.com/TERM/B/big_data.html}. 

	Přestože jednotná definice neexistuje, ustálilo se několik charakteristik, která mají Big Data společná. Jedná se o takzvaná 3+1V - Volume, Velocity, Variety a později přidaná vlastnost Veracity\cite{bakalarka Customer Intelligence v kontextu BigData}. Volume popisuje samotný objem zpracovávaných dat, Velocity pak rychlost, jakou data přibývají. Charakteristikou Variety popisujeme různorodost dat a Veracity určuje úplnost a důvěryhodnost dat. Tyto charakteristiky spíše popisuje klasické problémy, se kterými se s práci s Big Daty setkáme - data však nemusí splňovat všechny tyto vlastnosti zároveň, abychom je mohli za Big Data označit. 
	
	První technologie pro práci s Big Daty vznikaly na přelomu tisíciletí ve společnosti Google. Právě Google v roce 2004 zveřejnil článek o technologii MapReduce\cite{https://gigaom.com/2013/03/04/the-history-of-hadoop-from-4-nodes-to-the-future-of-data/}, která se stala stavebním kamenem pro většinu dalších technologií pro práci s velkými objemy dat. MapReduce vlastně popisuje dvě nezávislé funkce. První z nich je funkce Map, ve které jsou ze vstupních dat vygenerovány dvojice klíč a hodnota. Poté co je funkce Map dokončena, její výstup je použit jako vstup do funkce Reduce. Ta pak spojí vstupní data podle klíče\cite{https://www-01.ibm.com/software/data/infosphere/hadoop/mapreduce/}. Klíčovou vlastností MapReduce modelu je možnost paralelizace Map fáze na počítačovém clusteru. Jeden z počítačů v clusteru například přijme požadavek od uživatele. Tento počítač náhodně rozdělí vstupní data všem počítačům v clusteru a vyčká na provedení Map fáze jednotlivými počítači. Výsledná data pak sám master spojí v Reduce fázi a navrátí výsledek uživateli. 
	
	Hadoop
	
	Spark

%\section{Způsoby zpracování velkých objemů dat}
%	Batchové/streamové, proc se v mem pripade hodi streamove




\chapter{Analýza}
\section{Metody pro analýzu textu}
	Sentiment, ...

\section{Požadavky na systém}
	Funkční a nefunkční (co konkretne merit?, ...)

\section{Dostupné technologie}
	- dostupne technologie

\section{Twitter streaming API}
	Jak vypada API, tweet, ze API nenabizi vsechny prispevky - gnip.com

\chapter{Návrh}
\section{Celkový pohled na systém}
	jednotlive casti, propojeni, diagram 
\section{Analýza tweetů}
	jak bude vypadat program ve sparku
\section{Struktura databáze}
	schema, ...
\section{Návrh API}
	definice endpointů atd.

\chapter{Implementace}
\section{Použité technologie}
\subsection{Analýza proudu dat}
\subsection{Databáze}
\subsection{API webserver}
\subsection{Webová aplikace}

\chapter{Testování}
\section{Test API endpointů}
	da se web castecne povazovat jako otestovani api?
\section{???}

\chapter{Nasazení}

\chapter{Zhodnocení výsledků}



\begin{conclusion}
	Zaver
\end{conclusion}

% TODO: Promyslet, jak presne odevzdavat citace.
%\bibliographystyle{csn690}
%\bibliographystyle{iso690}
%\bibliography{ref}

\appendix

\chapter{Seznam použitých zkratek}
% \printglossaries
\begin{description}
	\item[Item1] foo
	\item[Item2] bar
\end{description}

\chapter{Obsah přiloženého CD}

%upravte podle skutecnosti

\begin{figure}
	\dirtree{%
		.1 readme.txt\DTcomment{stručný popis obsahu CD}.
		.1 exe\DTcomment{adresář se spustitelnou formou implementace}.
		.1 src.
		.2 impl\DTcomment{zdrojové kódy implementace}.
		.2 thesis\DTcomment{zdrojová forma práce ve formátu \LaTeX{}}.
		.1 text\DTcomment{text práce}.
		.2 thesis.pdf\DTcomment{text práce ve formátu PDF}.
		.2 thesis.ps\DTcomment{text práce ve formátu PS}.
	}
\end{figure}

\end{document}

\iffalse
\fi