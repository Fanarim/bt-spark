% options:
% thesis=B bachelor's thesis
% thesis=M master's thesis
% czech thesis in Czech language
% slovak thesis in Slovak language
% english thesis in English language
% hidelinks remove colour boxes around hyperlinks


\documentclass[thesis=B,czech]{FITthesis}[2012/06/26]

\usepackage[utf8]{inputenc} % LaTeX source encoded as UTF-8

\usepackage{graphicx} %graphics files inclusion
% \usepackage{amsmath} %advanced maths
% \usepackage{amssymb} %additional math symbols

\usepackage{dirtree} %directory tree visualisation



% % list of acronyms
% \usepackage[acronym,nonumberlist,toc,numberedsection=autolabel]{glossaries}
% \iflanguage{czech}{\renewcommand*{\acronymname}{Seznam pou{\v z}it{\' y}ch zkratek}}{}
% \makeglossaries

\newcommand{\tg}{\mathop{\mathrm{tg}}} %cesky tangens
\newcommand{\cotg}{\mathop{\mathrm{cotg}}} %cesky cotangens

% % % % % % % % % % % % % % % % % % % % % % % % % % % % % % 
% ODTUD DAL VSE ZMENTE
% % % % % % % % % % % % % % % % % % % % % % % % % % % % % % 

\department{Katedra softwarového inženýrství}
\title{Systém pro analýzu proudu dat v reálném čase}
\authorGN{David} %(křestní) jméno (jména) autora
\authorFN{Viktora} %příjmení autora
\authorWithDegrees{David Viktora} %jméno autora včetně současných akademických titulů
\supervisor{Ing. Adam Šenk}
\acknowledgements{Poděkování ....}
\abstractCS{V~několika větách shrňte obsah a přínos této práce v~češtině. Po přečtení abstraktu by se čtenář měl mít čtenář dost informací pro rozhodnutí, zda chce Vaši práci číst.}
\abstractEN{Sem doplňte ekvivalent abstraktu Vaší práce v~angličtině.}
\placeForDeclarationOfAuthenticity{V~Praze}
\declarationOfAuthenticityOption{4} %volba Prohlášení (číslo 1-6)
\keywordsCS{Nahraďte seznamem klíčových slov v češtině oddělených čárkou.}
\keywordsEN{Nahraďte seznamem klíčových slov v angličtině oddělených čárkou.}


\begin{document}

% \newacronym{CVUT}{{\v C}VUT}{{\v C}esk{\' e} vysok{\' e} u{\v c}en{\' i} technick{\' e} v Praze}
% \newacronym{FIT}{FIT}{Fakulta informa{\v c}n{\' i}ch technologi{\' i}}

\begin{introduction}
	Úvod práce
	
\end{introduction}

	\section{Cíle a požadavky práce}
	\section{Struktura práce}
%\chapter
%\section
%\subsection
%


\chapter{Úvod do problematiky}
\section{Big data}
\section{Způsoby zpracování velkých objemů dat}
\section{Dostupné technologie}

\chapter{Analýza}
\section{Funkční požadavky systému}
\section{Nefunkční požadavky}
\section{Twitter streaming API}

\chapter{Návrh}
\section{Celkový pohled na systém}
\section{Návrh API}
\section{Struktura databáze}

\chapter{Implementace}
\section{Použité technologie}
\subsection{Analýza proudu dat}
\subsection{Databáze}
\subsection{API webserver}
\subsection{Webová aplikace}

\chapter{Testování}
\section{Test API endpointů}
\section{???}

\chapter{Zhodnocení výsledků}


XXXXXXXXXXx \cite{JJ92}



\begin{conclusion}
	%sem napište závěr Vaší práce
\end{conclusion}

% TODO: Promyslet, jak presne odevzdavat citace.
%\bibliographystyle{csn690}
%\bibliographystyle{iso690}
%\bibliography{ref}

\appendix

\chapter{Seznam použitých zkratek}
% \printglossaries
\begin{description}
	\item[Item1] foo
	\item[Item2] bar
\end{description}

\chapter{Obsah přiloženého CD}

%upravte podle skutecnosti

\begin{figure}
	\dirtree{%
		.1 readme.txt\DTcomment{stručný popis obsahu CD}.
		.1 exe\DTcomment{adresář se spustitelnou formou implementace}.
		.1 src.
		.2 impl\DTcomment{zdrojové kódy implementace}.
		.2 thesis\DTcomment{zdrojová forma práce ve formátu \LaTeX{}}.
		.1 text\DTcomment{text práce}.
		.2 thesis.pdf\DTcomment{text práce ve formátu PDF}.
		.2 thesis.ps\DTcomment{text práce ve formátu PS}.
	}
\end{figure}

\end{document}

\iffalse
\fi